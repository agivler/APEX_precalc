\thispagestyle{empty}
\Huge
\noindent {\bf \textsc{Preface}}\\
\large
\emph{A Note on Using this Text}
\vspace{1in}

\normalsize

Thank you for taking your time to read this preface. We will briefly share some key features of this text to (hopefully) improve your experience using it.\\

\noindent\textbf{\large For Instructors: How to Use this Text}\\

This text was written as a prequel to the \apex Calculus series, a three--volume series on Calculus. This text is not intended to fully prepare students with all of the mathematical knowledge they need to tackle Calculus, rather it is designed to review mathematical concepts that are often stumbling blocks in the Calculus sequence. It starts basic and builds to more complex topics. This text is written so that each section and topic largely stands on its own, making it a good resource for students in Calculus who are struggling with the supporting mathematics found in Calculus courses. The topics were chosen based on experience; several instructors in the Applied Mathematics Department at the Virginia Military Institute (VMI) compiled a list of topics that Calculus students commonly struggle with, giving the focus of this text. This allows for a more focused approach; at first glance one of the obvious differences from a standard Pre-Calculus text is its size. 

This text, as well as the three volumes of the \apex Calculus series, is available separately for free at \texttt{\href{http://apexcalculus.com}{www.apexcalculus.com}}. All four texts can be purchased as bound volumes for \$15 or less per text at \href{http://amazon.com}{Amazon.com}.\\ 



%A result of this splitting is that sometimes a concept is said to be explored in a ``later section,'' though that section does not actually appear in this particular text. Downloading the .pdf of \textit{APEX Calculus} will ensure that you have all the content.  
%material is referenced that is not contained in the present text. The context should make it clear whether the ``missing'' material is in the \textit{Calculus I} or \textit{Calculus III} portion. Downloading the appropriate .pdf, or the whole \textit{APEX Calculus} .pdf, will give access to these topics.
% This splitting of the material also results in unfortunate page/chapter numberings. Chapter 5 of this text is Chapter 1 of \textit{Calculus II}. Apart from these numberings, page--for--page the content of the sections that appear in both \textit{Calculus I} and \textit{Calculus II} are identical.\\ %For instance, in this text, ``Theorem 20'' may be mentioned, although Theorem 20 is only presented in Part I. To minimize confusion, theorems, definitions and key ideas are referenced by their title or subject matter, not their number.

%The current publisher of this text does not allow one text to be split across multiple volumes, with continuity of chapters and page numberings. This is the one drawback of the current publishing model that has many advantages, highlighted below. Because of this, there are a few peculiarities 

\noindent\textbf{\large For Students: How to Read this Text}\\

Many mathematical texts are written in very formal, succinct language. This is a terrific approach if this text is simply used as a resource for someone who is already comfortable with the ideas in the text, but can make it difficult for anyone who is new to the material. This text was written in a different fashion. Its goal is to show you mathematical ideas and concepts explained in an informal style so that your focus is on learning the math, not trying to decipher the sentences.

This text is written with many examples. Each new idea is shown through several examples, starting with a straightforward example and working up to more complex examples of the idea. These examples, and the exercises in the text, come from the mathematics as it appears in Calculus. You may notice that if you use this text as a resource while you are taking Calculus that many of the problems in the text come from problems in Calculus. For example, many of the questions asking you to simplify a function are really unsimplified derivatives of common function types, a type of function with a special meaning that is used in Calculus. 

Additionally, the later sections of this text will reinforce many of the ideas of the earlier sections. This is entirely on purpose. In Calculus, you will need to use many of these skills in the solution of a larger problem. The larger problems almost never tell you the names of the skills you will use. This means that you need to identify which skills to use and when to use them. To help get you used to these types of problems, this text often gives you problems that require skills from earlier sections, without telling you about it.

Finally, answers (but not solutions) to all exercises are provided in an appendix at the end of the text. We highly recommend checking your answer to each exercise before moving on to another exercise. This will prevent you from practicing skills incorrectly and will save you the potential frustration of finding out that you have done several problems and made the same mistake on all of them. \\

\noindent\textbf{\large Thanks}\\

Many people contributed to this text, in ways small and large. First, thanks are due to the VMI students who first used this text during the Summer Transition Program (STP) in 2017 and the VMI cadets who used it during Fall 2017. These students were diligent readers who found many typos and made suggestions that greatly improved the usefulness of the text for students. Second, thanks to Meagan Herald, not only for proofreading the entire text and answer key multiple times, but for never complaining about using a work in progress for the basis of VMI's Pre-Calculus course. Major revisions were made between STP 2017 and Fall 2017 with her help and guidance, including reordering the sections of the text so that exercises and examples did not use concepts that had not been discussed by that point. Meagan also initially developed a list of topics that formed the basis for those used in this text and well as the Pre-Calculus course at VMI.

Additionally, I would like to thank Jessica Libertini for motivating me to complete this text and providing me with many of the problems used in the text as well as pedagogical advice. Furthermore, the hard work of Greg Hartman who authored and maintains the \apex Calculus series needs to be acknowledged; he is responsible for creating the formatting files that hold this text together. Throughout this process, we were also given large amounts of support by the Applied Mathematics Department at the Virginia Military Institute, most notably the department head, Troy Siemers who supported our efforts from the beginning.

Finally, thanks are due to my husband Jonathan Chapman who supported me as I worked on this text extra hours outside of the office and who provided me with technical support to streamline the creation process. \\

\noindent\textbf{\large \apex\  -- Affordable Print and Electronic teXts}\\

\apex\ is a consortium of authors  who collaborate to produce high--quality, low--cost textbooks. The current textbook--writing paradigm is facing a potential revolution as desktop publishing and electronic formats increase in popularity. However, writing a good textbook is no easy task, as the time requirements alone are substantial. It takes countless hours of work to produce text, write examples and exercises, edit and publish. Through collaboration, however, the cost to any individual can be lessened, allowing us to create texts that we freely distribute electronically and sell in printed form for an incredibly low cost. 

Each text is available as a free .pdf, protected by a Creative Commons Attribution - Noncommercial 4.0 copyright. That  means you can give the .pdf to anyone you like, print it in any form you like, and even edit the original content and redistribute it. If you do the latter, you must  clearly reference this work and you cannot sell your edited work for money.

We encourage others to adapt this work to fit their own needs. One might add sections that are ``missing'' or remove sections that your students won't need. The source files can be found at \texttt{\href{https://github.com/APEXCalculus}{github.com/APEXCalculus}}.

You can learn more at \texttt{\href{http://www.vmi.edu/APEX}{www.vmi.edu/APEX}}.
\thispagestyle{empty}

